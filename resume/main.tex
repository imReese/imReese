\documentclass{resume}

\usepackage[T1]{fontenc}
\usepackage{lmodern}

\name{ 段冰 }
\phone{ +86 18510840710 }
\email{ reese\_duan@163.com }   % _在latex中表示角标 注意使用\反义
\link{ https://github.com/imReese }
\linkname{ Github/imReese }   % 可以自定义也可以同\link
\photo{ reese.png }

\begin{document}
\begin{CJK}{UTF8}{gbsn}

\noindent\textbf{求职目标:软件开发工程师或后端开发工程师}

\begin{EducationSection}
  \begineducationitem
    { 南京大学 }
    { 2017.09 -- 2021.06 }
    { 学士 环境工程}
    {GPA: 4.2/5.0 }
    \item 核心课程: 计算机组成原理、数据结构、操作系统、计算机网络、计算机体系结构
    \item 相关课程: 线性代数、概率论与数理统计、算法设计与分析
  \endeducationitem
\end{EducationSection}

\begin{WorkExperienceSection}
  \beginworkitem
    { 华为云计算公司 - 竖亥lab }
    { 2023.07 - 2024.07 }
    { 计算体系结构研究工程师 }
    \item \textbf{研究方向:CPU微架构性能优化与指令流分析}
    \item \textbf{微架构指标采集工具开发:}基于Perf及Intel Pin工具链,设计并实现低开销指令流采集工具,覆盖MySQL、Redis等云服务关键组件,支持每秒百万级指令采样,为性能瓶颈分析提供核心数据支撑。
    \item \textbf{TLB调度策略创新:}通过指令流特征分析,提出新型TLB动态调度算法,在ChampSim仿真器中实现并验证,单核场景tlb命中率提高30\%,IPC提高6\%,相关方案已申请发明专利。
    \item \textbf{跨团队协作优化验证:}协助海思、欧拉团队进行SMT多线程优化策略在鲲鹏新一代 CPU 的落地验证,通过指令级并行度分析推动硬件微码更新,单核吞吐量提升8\%。
  \endworkitem

  \beginworkitem
    { 华为数据存储产品线 - 基础设施开发部 }
    { 2022.05 - 2023.07 }
    { 软件开发工程师 }
    \item \textbf{核心领域:分布式系统高可用与集群管理}
    \item \textbf{集群管理模块模块功能迭代:}负责基于Paxos算法的分布式多节点集群管理模块,包括自选主及客户端选主
    \item \textbf{集群规模扩展优化:}重构集群管理组件通信模型,通过异步消息队列+模块间进一步解耦合,单集群支持节点数从128提升至256,资源初始化效率提升40\%。
    \item \textbf{高可靠性与安全性加固:}\par
    \hspace{1.8em} 主导模块白盒测试体系搭建,基于 DTFuzz 框架完成200+接口自动化测试,分支覆盖率90\%+,提前拦截5类潜在缺陷;\par
    \hspace{1.8em} 设计模块间参数安全拦截机制,通过输入过滤 + 沙箱隔离阻断命令注入攻击,漏洞风险降低70\%;\par
    \hspace{1.8em} 解决20+现网高优先级问题并发布补丁版本,现网故障率下降35\%。\par
  \endworkitem
\end{WorkExperienceSection}

\begin{ProjectSection}
  \beginprojectitem
    { Leetcode 自动化管理刷题工具 }
    { C++, python3 }
    { https://github.com/imReese/leetcode-solutions }
    { 2025.02 -- 至今 }
    \item 实现 Leetcode 题目自动解析,自动生成题目描述、代码、测试用例
    \item 结合 Github Actions 实现解题数据更新并生成导航页面
  \endprojectitem

  \beginprojectitem
    { C++ 轻量级 Web 服务器 }
    { C++, C }
    { https://github.com/imReese/tiny-webserver }
    { 2021.12 -- 2022.03 }
    \item 使用线程池 + 非阻塞 socket + epoll + 事件处理的并发模型
    \item 使用状态机解析 HTTP 请求报文,支持解析 GET 和 POST 请求
    \item 访问服务器数据库实现 web 端用户注册、登录功能,可以请求服务器图片和视频文件
    \item 实现同步/异步日志系统,记录服务器运行状态
  \endprojectitem

\end{ProjectSection}

\begin{SkillsSection}
  { 熟练掌握数据结构与算法,熟悉 Git 版本控制工具和 Linux 命令}
  \item \textbf{ 编程语言:} C/C++, python3, shell, LaTeX
  \item \textbf{ 工具框架:} Linux, Git, Docker, perf, gdb, vim, Makefile, CMake
\end{SkillsSection}

\begin{InterestsSection}
  { 马拉松、健身、篮球、网球 }
\end{InterestsSection}

\end{CJK}
\end{document}
